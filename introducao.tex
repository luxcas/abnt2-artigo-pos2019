%\section{Proposta}
%Turma: Pós JK 2º/2019 - Especialização em Gestão e Desenvolvimento de Sistemas de Informação
%
%
%
%Nome do aluno: Lucas Carvalho de Aquino
%
%
%
%Assunto: Sustentação de sistemas com metodologia ágil 
%
%
%
%Título: Utilizando metodologia ágil para sustententação de Portais 
%
%Problema:  Após a implantação de portais para internet, surgem várias novas implementações de funcionalidades e correções. Com isso a equipe de desenvolvimento fica, muitas vezes, atrapalhada sem saber o que atender primeiro.
%
%Solução: Propor melhorias nos processos de desenvolvimento utilizando metodologia agil, prevendo as interferencias do Product Owner para atendimento de demandas/implementações em funcionalidades já disponíveis ou erros que estão ocorrendo em produção.
%
%Referências 
%
%Scrum  - https://www.scrumguides.org/docs/scrumguide/v1/Scrum-Guide-Portuguese-BR.pdf
%
%Scrum/ITIL  - https://pt.slideshare.net/Exin/scrum-no-gerenciamento-de-problemas-e-incidentes-itil
%
%ECM Enterprise Content Management, Ulrich Kampffmeyer. Hamburg 2006, ISBN 978-3-936534-09-8. Definitions, Scope, Architecture, Components and ECM-Suites in English, French, and German - https://www.project-consult.de/Files/ECM\_White\%20Paper\_kff\_2006.pdf
%
%ITIL - MAGALHÃES, Ivan L.; PINHEIRO, Walfrido B. Gerenciamento de Serviços de TI na Prática: Uma abordagem com base na ITIL. 1a Ed. São Paulo: Novatec, 2007.
%
%ITIL - https://www.axelos.com/best-practice-solutions/itil/what-is-itil
%
%ITIL - AXELOS (2019). «5.2 Service management practices». ITIL Foundation, ITIL 4 edition. [S.l.]: TSO (The Stationery Office)
%
%Scrumban - https://www.agilealliance.org/what-is-scrumban/


\section*{Introdução}

As práticas agéis de apoiam no desenvolvimento, na melhoria e na sustentação de sistemas. baseando-se nos principios do manifesto ágil onde existirão mudanças nos requisitos das funcionalidades entregues e que serão importantes para o usuário final.
No 


