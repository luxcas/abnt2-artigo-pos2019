%% abtex2-modelo-artigo.tex, v-1.7.1 laurocesar
%% Copyright 2012-2013 by abnTeX2 group at http://abntex2.googlecode.com/ 
%%
%% This work may be distributed and/or modified under the
%% conditions of the LaTeX Project Public License, either version 1.3
%% of this license or (at your option) any later version.
%% The latest version of this license is in
%%   http://www.latex-project.org/lppl.txt
%% and version 1.3 or later is part of all distributions of LaTeX
%% version 2005/12/01 or later.
%%
%% This work has the LPPL maintenance status `maintained'.
%% 
%% The Current Maintainer of this work is the abnTeX2 team, led
%% by Lauro César Araujo. Further information are available on 
%% http://abntex2.googlecode.com/
%%
%% This work consists of the files abntex2-modelo-artigo.tex and
%% abntex2-modelo-references.bib
%%

% ------------------------------------------------------------------------
% ------------------------------------------------------------------------
% abnTeX2: Modelo de Artigo Acadêmico em conformidade com
% ABNT NBR 6022:2003: Informação e documentação - Artigo em publicação 
% periódica científica impressa - Apresentação
% ------------------------------------------------------------------------
% ------------------------------------------------------------------------

\documentclass[
	% -- opções da classe memoir --
	article,			% indica que é um artigo acadêmico
	11pt,				% tamanho da fonte
	oneside,			% para impressão apenas no verso. Oposto a twoside
	a4paper,			% tamanho do papel. 
	% -- opções da classe abntex2 --
	%chapter=TITLE,		% títulos de capítulos convertidos em letras maiúsculas
	%section=TITLE,		% títulos de seções convertidos em letras maiúsculas
	%subsection=TITLE,	% títulos de subseções convertidos em letras maiúsculas
	%subsubsection=TITLE % títulos de subsubseções convertidos em letras maiúsculas
	% -- opções do pacote babel --
	english,			% idioma adicional para hifenização
	brazil,				% o último idioma é o principal do documento
	]{abntex2}


% ---
% PACOTES
% ---

% ---
% Pacotes fundamentais 
% ---
\usepackage{cmap}				% Mapear caracteres especiais no PDF
\usepackage{lmodern}			% Usa a fonte Latin Modern
\usepackage[T1]{fontenc}		% Selecao de codigos de fonte.
\usepackage[utf8]{inputenc}		% Codificacao do documento (conversão automática dos acentos)
\usepackage{indentfirst}		% Indenta o primeiro parágrafo de cada seção.
\usepackage{nomencl} 			% Lista de simbolos
\usepackage{color}				% Controle das cores
\usepackage{graphicx}			% Inclusão de gráficos
% ---
		
% ---
% Pacotes adicionais, usados apenas no âmbito do Modelo Canônico do abnteX2
% ---
\usepackage{lipsum}				% para geração de dummy text
% ---
		
% ---
% Pacotes de citações
% ---
\usepackage[brazilian,hyperpageref]{backref}	 % Paginas com as citações na bibl
\usepackage[alf]{abntex2cite}	% Citações padrão ABNT
% ---

% ---
% Configurações do pacote backref
% Usado sem a opção hyperpageref de backref
\renewcommand{\backrefpagesname}{Citado na(s) página(s):~}
% Texto padrão antes do número das páginas
\renewcommand{\backref}{}
% Define os textos da citação
\renewcommand*{\backrefalt}[4]{
	\ifcase #1 %
		Nenhuma citação no texto.%
	\or
		Citado na página #2.%
	\else
		Citado #1 vezes nas páginas #2.%
	\fi}%
% ---

% ---
% Informações de dados para CAPA e FOLHA DE ROSTO
% ---
\titulo{Utilizando metodologia ágil para sustententação de Portais}
\autor{Lucas Carvalho de Aquino\thanks{lucas@lucasaquino.com.br} }
\local{Brasil}
\data{Pós JK 2º/2019 - Especialização em Gestão e Desenvolvimento de Sistemas de Informação\
  SGAS Quadra 616 Módulo 114, Asa Sul - Brasília/DF - CEP 70.200-760 \\ 21 de Janeiro de 2020, v 1.0}
% ---

% ---
% Configurações de aparência do PDF final

% alterando o aspecto da cor azul
\definecolor{blue}{RGB}{41,5,195}

% informações do PDF
\makeatletter
\hypersetup{pdftitle={\@title}, 
		pdfauthor={\@author},
    	pdfsubject={Utilizando metodologia ágil para sustententação de Portais},
	    pdfcreator={Lucas Carvalho de Aquino},
		pdfkeywords={abnt}{latex}{abntex}{abntex2}{atigo científico}, 
		colorlinks=False
	}
\makeatother
% --- 

% ---
% compila o indice
% ---
\makeindex
% ---

% ---
% Altera as margens padrões
% ---
\setlrmarginsandblock{3cm}{3cm}{*}
\setulmarginsandblock{3cm}{3cm}{*}
\checkandfixthelayout
% ---

% --- 
% Espaçamentos entre linhas e parágrafos 
% --- 

% O tamanho do parágrafo é dado por:
\setlength{\parindent}{1.3cm}

% Controle do espaçamento entre um parágrafo e outro:
\setlength{\parskip}{0.2cm}  % tente também \onelineskip

% Espaçamento simples
\SingleSpacing

% ----
% Início do documento
% ----
\begin{document}
% Retira espaço extra obsoleto entre as frases.
\frenchspacing 

% ----------------------------------------------------------
% ELEMENTOS PRÉ-TEXTUAIS
% ----------------------------------------------------------

%---
%
% Se desejar escrever o artigo em duas colunas, descomente a linha abaixo
% e a linha com o texto ``FIM DE ARTIGO EM DUAS COLUNAS''.
% \twocolumn[    		% INICIO DE ARTIGO EM DUAS COLUNAS
%
%---
% página de titulo

\renewcommand{\resumoname}{Abstract}
\begin{resumoumacoluna}
 \begin{otherlanguage*}{english}
	This article deals with the use of agile methodology for sustaining Portals, with auxiliary tools in the planning of meeting the requests for improvements made by the Project Owner or analyzed by the management committee. Along with the project's evolution requests, there is also the fulfillment of the demands generated to correct errors in the content management system that can be visual problems in the layout (HTM, CSS or JS) or backend with both queries in the database without the expected result or unexpected errors presented by the system.

   \vspace{\onelineskip}

   \noindent
   \textbf{Key-words}: ITIL, Internet portals, Support, Scrum, Kanban.
 \end{otherlanguage*}  
\end{resumoumacoluna}

% resumo em português
\renewcommand{\resumoname}{Resumo}
\begin{resumoumacoluna}
  Este artigo trata da utilização de metodologia ágil para sustententação de Portais, com ferramentas auxiliares no planejamento de atendimento às solicitações de melhorias feitas pelo Project Owner ou analisadas pelo comitê gestor. Junto com as solicitações de evolução do projeto existem tambem o atendimento das demandas geradas para correção de erros no sistema de gerenciamento de conteúdos que podem ser problemas visuais no layout(HTM, CSS ou JS) ou de backend tanto com consultas no banco de dados sem o resultado esperado ou erros inesperados apresentados pelo sistema.
  
\vspace{\onelineskip}

\noindent \textbf{Palavras-Chave}: ITIL, Portais para internet, Sustentação, Scrum, Kanban.
\end{resumoumacoluna}

% ]  				
% FIM DE ARTIGO EM DUAS COLUNAS
% ---

% ----------------------------------------------------------
% ELEMENTOS TEXTUAIS
% ----------------------------------------------------------
\textual

% ----------------------------------------------------------
% Introdução
% ----------------------------------------------------------

%\section{Proposta}
%Turma: Pós JK 2º/2019 - Especialização em Gestão e Desenvolvimento de Sistemas de Informação
%
%
%
%Nome do aluno: Lucas Carvalho de Aquino
%
%
%
%Assunto: Sustentação de sistemas com metodologia ágil 
%
%
%
%Título: Utilizando metodologia ágil para sustententação de Portais 
%
%Problema:  Após a implantação de portais para internet, surgem várias novas implementações de funcionalidades e correções. Com isso a equipe de desenvolvimento fica, muitas vezes, atrapalhada sem saber o que atender primeiro.
%
%Solução: Propor melhorias nos processos de desenvolvimento utilizando metodologia agil, prevendo as interferencias do Product Owner para atendimento de demandas/implementações em funcionalidades já disponíveis ou erros que estão ocorrendo em produção.
%
%Referências 
%
%Scrum  - https://www.scrumguides.org/docs/scrumguide/v1/Scrum-Guide-Portuguese-BR.pdf
%
%Scrum/ITIL  - https://pt.slideshare.net/Exin/scrum-no-gerenciamento-de-problemas-e-incidentes-itil
%
%ECM Enterprise Content Management, Ulrich Kampffmeyer. Hamburg 2006, ISBN 978-3-936534-09-8. Definitions, Scope, Architecture, Components and ECM-Suites in English, French, and German - https://www.project-consult.de/Files/ECM\_White\%20Paper\_kff\_2006.pdf
%
%ITIL - MAGALHÃES, Ivan L.; PINHEIRO, Walfrido B. Gerenciamento de Serviços de TI na Prática: Uma abordagem com base na ITIL. 1a Ed. São Paulo: Novatec, 2007.
%
%ITIL - https://www.axelos.com/best-practice-solutions/itil/what-is-itil
%
%ITIL - AXELOS (2019). «5.2 Service management practices». ITIL Foundation, ITIL 4 edition. [S.l.]: TSO (The Stationery Office)
%
%Scrumban - https://www.agilealliance.org/what-is-scrumban/


\section*{Introdução}

As práticas agéis de apoiam no desenvolvimento, na melhoria e na sustentação de sistemas. baseando-se nos principios do manifesto ágil onde existirão mudanças nos requisitos das funcionalidades entregues e que serão importantes para o usuário final.
No 




\section{Ferramentas , metodologias e frameworks} \label{ses:intro}

Nesta seção será apresentada algumas ferramentas, metologias e frameworks que estão inseridos no processo de sustentação de portais de internet.

\subsection{Plone CMS}

Plone CMS é um sistema de gerenciamento de conteúdo integrado ao servidor de aplicações Zope \cite{ZopeF} baseados em Python para criar aplicativos Web seguros e altamente escaláveis. além de linguagem de templates própria. \cite{PloneCMS}

O Plone pode ser usado na  construção de portais de internet, extranets ou intranets. Podendo ser utilizado para construção de diversos sistemas de informação, tais como, gerenciamento de documentos, formulários de captação de dados, ferramenta para trabalho colaborativo e etc. 

\subsection{SCRUM}

Scrum é uma estrutura de processo usada para gerenciar o trabalho em produtos complexos desde o início dos anos 90. Scrum não é um processo, técnica ou método definitivo. Pelo contrário, é uma estrutura na qual você pode empregar vários processos e técnicas. \cite{Scrum}

O Scrum deixa clara a eficácia relativa do gerenciamento de produtos e das técnicas de trabalho para que você possa melhorar continuamente o produto, a equipe e o ambiente de trabalho.

A estrutura do Scrum consiste em equipes do Scrum e suas funções, eventos, artefatos e regras associados. Cada componente da estrutura serve a um propósito específico e é essencial para o sucesso e uso do Scrum.

As regras do Scrum unem os papéis, eventos e artefatos, governando os relacionamentos e a interação entre eles.

\subsection{KANBAN}

Kanban (subs):  uma estratégia para otimizar o fluxo de valor para stakeholders através de um processo que utiliza um sistema visual que limita a quantidade de trabalho em andamento através de um sistema puxado. O conceito de “fluxo” é primordial para a definição do Kanban. Fluxo é o movimento de valor para o  cliente  através  do  sistema  de  desenvolvimento  de  Produtos.  O  Kanban  otimiza  o  fluxo  ao melhorar a eficiência, eficácia e previsibilidade gerais de um processo.\cite{Kanban}

O Kanban também é uma ferramenta visual criada com o objetivo de manter o alto nível de produção no desenvolvimento de seus sistemas. Sua natureza altamente visual permite que as equipes se comuniquem mais facilmente sobre suas metas e prazos. Representado por um quadro, geralmente, utiliza cards para representar um fluxo pré-estabelecido das etapas de um processo. Esse sistema também busca identificar e resolver restrições que limitam sua performance.

\subsection{ITIL}

O ITIL é uma abordagem amplamente aceita para gerenciamento de serviços de TI (ITSM), adotada por indivíduos e organizações em todo o mundo. Ele fornece um conjunto coeso de melhores práticas, retirado dos setores público e privado internacionalmente.\cite{ITILLivroAXELOS} |\cite{ITIL}

ITIL é uma biblioteca de publicações a respeito das melhores práticas de gestão de serviços de TI. Esses materiais contêm informações sobre funções e processos a respeito, serviços web, nuvem, outsourcing e muito mais. Utilizamos essa metodologia para melhorar nossos serviços maximizando os benefícios para as organizações, fornecendo cada vez mais valor. Essa estratégia de promover contínuas melhorias nos permite planejar e gerenciar as mudanças necessárias nos sistemas ou implantar novas estruturas, dessa forma, diminuímos os riscos de falhas ou interrupções.

Dentre os livros utilizados no gerrenciamento de serviços de TI, o gerenciamento de incidentes é adotados para que os incidentes ocorridos no sistema sejam resolvidos o mais breve possível, não comprometendo o Acordo de Nivel de Serviço celebrado com a área cliente do serviço de TI. \cite{ITILLivro}

\subsection{Redmine}
O Redmine é um aplicativo da web de gerenciamento de projetos flexível. Escrito usando a estrutura Ruby on Rails, é multiplataforma e entre bancos de dados.\cite{Remine}







%\section{Trabalhos Relacionados} \label{ses:rela}



 
 
	
\section{Cenário Atual}\label{aeb}

A Empresa X  a área de comunicação social - ASCOM é responsavel pela manutenção das páginas e aplicativos web , tornando-se o principal cliente do Portal de internet. As decisões de implementação de melhorias ou novas funcionalidades primeiramente passam pele cliente antes de chegar na área de TI responsável por desenvolver as novas  funcionalidades e correção de erros no sistema por trás do Portal de internet, o Plone CMS\cite{PloneCMS}.

Desde a entrega do Portal de internet em produção, os chamados referente aos erros que os usuários internos e externos encontravam no sistema são cadastrados pelo service desk no sistema, caso quem encontra-se o erro fosse a ASCOM o erro era repassado diretamente para área de TI as vezes por email em outras era cadastrado no sistema de chamados.

Nas melhorias e novas funcionalidades eram encaminhadas por email pra área de TI, que organizava a ordem de chegada priorizando as mais importantes sinalizadas pela ASCOM.

A rotina normal de atendimento partia da fila de chamados, por ordem de chegada, os chamados eram atendidos. E as melhorias e novas funcionalidades seguiam a priorização que a ASCOM definia de acordo com os interesses dela.

Nesse cenário ocorriam problemas de demora no atendimento de melhorias, pois a equipe estava ocupada com a correção de erros, ou quando a equipe de TI estava no atendimento de melhorias, buscando novas soluções existia a espera na correção de erros causando frustração nos usuários que não conseguiam utilizar os serviços do portal de internet .




\section{Proposta de melhoria do processo de sustentação}\label{sec:proposta}

Detalhar melhoria



\subsection{Scrumban}

Falta

\subsection{Sprints}

Falta

\input{resultado}

\input{conclusao}

% ---
% Finaliza a parte no bookmark do PDF, para que se inicie o bookmark na raiz
% ---
\bookmarksetup{startatroot}% 
% ---

% ---
% Conclusão
% ---

% ----------------------------------------------------------
% Referências bibliográficas
% ----------------------------------------------------------
\bibliography{sbc-template}


\end{document}
