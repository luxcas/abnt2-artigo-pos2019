\section{Ferramentas , metodologias e frameworks} \label{ses:intro}

Nesta seção será apresentada algumas ferramentas, metologias e frameworks que estão inseridos no processo de sustentação de portais de internet.

\subsection{Plone CMS}

Plone CMS é um sistema de gerenciamento de conteúdo integrado ao servidor de aplicações Zope \cite{ZopeF} baseados em Python para criar aplicativos Web seguros e altamente escaláveis. além de linguagem de templates própria. \cite{PloneCMS}

O Plone pode ser usado na  construção de portais de internet, extranets ou intranets. Podendo ser utilizado para construção de diversos sistemas de informação, tais como, gerenciamento de documentos, formulários de captação de dados, ferramenta para trabalho colaborativo e etc. 

\subsection{SCRUM}

Scrum é uma estrutura de processo usada para gerenciar o trabalho em produtos complexos desde o início dos anos 90. Scrum não é um processo, técnica ou método definitivo. Pelo contrário, é uma estrutura na qual você pode empregar vários processos e técnicas. \cite{Scrum}

O Scrum deixa clara a eficácia relativa do gerenciamento de produtos e das técnicas de trabalho para que você possa melhorar continuamente o produto, a equipe e o ambiente de trabalho.

A estrutura do Scrum consiste em equipes do Scrum e suas funções, eventos, artefatos e regras associados. Cada componente da estrutura serve a um propósito específico e é essencial para o sucesso e uso do Scrum.

As regras do Scrum unem os papéis, eventos e artefatos, governando os relacionamentos e a interação entre eles.

\subsection{KANBAN}

Kanban (subs):  uma estratégia para otimizar o fluxo de valor para stakeholders através de um processo que utiliza um sistema visual que limita a quantidade de trabalho em andamento através de um sistema puxado. O conceito de “fluxo” é primordial para a definição do Kanban. Fluxo é o movimento de valor para o  cliente  através  do  sistema  de  desenvolvimento  de  Produtos.  O  Kanban  otimiza  o  fluxo  ao melhorar a eficiência, eficácia e previsibilidade gerais de um processo.\cite{Kanban}

O Kanban também é uma ferramenta visual criada com o objetivo de manter o alto nível de produção no desenvolvimento de seus sistemas. Sua natureza altamente visual permite que as equipes se comuniquem mais facilmente sobre suas metas e prazos. Representado por um quadro, geralmente, utiliza cards para representar um fluxo pré-estabelecido das etapas de um processo. Esse sistema também busca identificar e resolver restrições que limitam sua performance.

\subsection{ITIL}

O ITIL é uma abordagem amplamente aceita para gerenciamento de serviços de TI (ITSM), adotada por indivíduos e organizações em todo o mundo. Ele fornece um conjunto coeso de melhores práticas, retirado dos setores público e privado internacionalmente.\cite{ITILLivroAXELOS}\cite{ITIL}

ITIL é uma biblioteca de publicações a respeito das melhores práticas de gestão de serviços de TI. Esses materiais contêm informações sobre funções e processos a respeito, serviços web, nuvem, outsourcing e muito mais. Utilizamos essa metodologia para melhorar nossos serviços maximizando os benefícios para as organizações, fornecendo cada vez mais valor. Essa estratégia de promover contínuas melhorias nos permite planejar e gerenciar as mudanças necessárias nos sistemas ou implantar novas estruturas, dessa forma, diminuímos os riscos de falhas ou interrupções.

Dentre os livros utilizados no gerrenciamento de serviços de TI, o gerenciamento de incidentes é adotados para que os incidentes ocorridos no sistema sejam resolvidos o mais breve possível, não comprometendo o Acordo de Nivel de Serviço celebrado com a área cliente do serviço de TI. \cite{ITILLivro}

\subsection{Redmine}
O Redmine é um aplicativo da web de gerenciamento de projetos flexível. Escrito usando a estrutura Ruby on Rails, é multiplataforma e entre bancos de dados.\cite{Remine}





