\section{Cenário Atual}\label{aeb}

A Empresa X  a área de comunicação social - ASCOM é responsavel pela manutenção das páginas e aplicativos web , tornando-se o principal cliente do Portal de internet. As decisões de implementação de melhorias ou novas funcionalidades primeiramente passam pele cliente antes de chegar na área de TI responsável por desenvolver as novas  funcionalidades e correção de erros no sistema por trás do Portal de internet, o Plone CMS\cite{PloneCMS}.

Desde a entrega do Portal de internet em produção, os chamados referente aos erros que os usuários internos e externos encontravam no sistema são cadastrados pelo service desk no sistema, caso quem encontra-se o erro fosse a ASCOM o erro era repassado diretamente para área de TI as vezes por email em outras era cadastrado no sistema de chamados.

Nas melhorias e novas funcionalidades eram encaminhadas por email pra área de TI, que organizava a ordem de chegada priorizando as mais importantes sinalizadas pela ASCOM.

A rotina normal de atendimento partia da fila de chamados, por ordem de chegada, os chamados eram atendidos. E as melhorias e novas funcionalidades seguiam a priorização que a ASCOM definia de acordo com os interesses dela.

Nesse cenário ocorriam problemas de demora no atendimento de melhorias, pois a equipe estava ocupada com a correção de erros, ou quando a equipe de TI estava no atendimento de melhorias, buscando novas soluções existia a espera na correção de erros causando frustração nos usuários que não conseguiam utilizar os serviços do portal de internet .


